% --- Template for thesis / report with tktltiki2 class ---
%
% last updated 2013/02/15 for tkltiki2 v1.02

\documentclass[english]{tktltiki2}

% tktltiki2 automatically loads babel, so you can simply
% give the language parameter (e.g. finnish, swedish, english, british) as
% a parameter for the class: \documentclass[finnish]{tktltiki2}.
% The information on title and abstract is generated automatically depending on
% the language, see below if you need to change any of these manually.
%
% Class options:
% - grading                 -- Print labels for grading information on the front page.
% - disablelastpagecounter  -- Disables the automatic generation of page number information
%                              in the abstract. See also \numberofpagesinformation{} command below.
%
% The class also respects the following options of article class:
%   10pt, 11pt, 12pt, final, draft, oneside, twoside,
%   openright, openany, onecolumn, twocolumn, leqno, fleqn
%
% The default font size is 11pt. The paper size used is A4, other sizes are not supported.
%
% rubber: module pdftex

% --- General packages ---

\usepackage[utf8]{inputenc}
\usepackage[T1]{fontenc}
\usepackage{lmodern}
\usepackage{microtype}
\usepackage{amsfonts,amsmath,amssymb,amsthm,booktabs,color,enumitem,graphicx}
\usepackage[pdftex,hidelinks]{hyperref}

% Automatically set the PDF metadata fields
\makeatletter
\AtBeginDocument{\hypersetup{pdftitle = {\@title}, pdfauthor = {\@author}}}
\makeatother

% --- Language-related settings ---
%
% these should be modified according to your language

% babelbib for non-english bibliography using bibtex
\usepackage[fixlanguage]{babelbib}

% add bibliography to the table of contents
\usepackage[nottoc]{tocbibind}

% --- Theorem environment definitions ---

\newtheorem{thm}{Theorem}
\newtheorem{lem}[thm]{Lemma}
\newtheorem{cor}[thm]{Corollary}

\theoremstyle{definition}
\newtheorem{definition}[thm]{Definition}

\theoremstyle{remark}
\newtheorem*{remark}{Remark}


% --- tktltiki2 options ---
%
% The following commands define the information used to generate title and
% abstract pages. The following entries should be always specified:

\title{Effects of design methods in achieving shared understanding}
\author{Mika Kivi}
\date{\today}
\level{Master Thesis}
\abstract{Abstract.}

% The following can be used to specify keywords and classification of the paper:

\keywords{Software development, design, shared understanding, method}

% classification according to ACM Computing Classification System (http://www.acm.org/about/class/)
% This is probably mostly relevant for computer scientists
% uncomment the following; contents of \classification will be printed under the abstract with a title
% "ACM Computing Classification System (CCS):"
% \classification{}

% If the automatic page number counting is not working as desired in your case,
% uncomment the following to manually set the number of pages displayed in the abstract page:
%
% \numberofpagesinformation{16 pages + 10 appendix pages}
%
% If you are not a computer scientist, you will want to uncomment the following by hand and specify
% your department, faculty and subject by hand:
%
% \faculty{Faculty of Science}
% \department{Department of Computer Science}
% \subject{Computer Science}
%
% If you are not from the University of Helsinki, then you will most likely want to set these also:
%
% \university{University of Helsinki}
% \universitylong{HELSINGIN YLIOPISTO --- HELSINGFORS UNIVERSITET --- UNIVERSITY OF HELSINKI} % displayed on the top of the abstract page
% \city{Helsinki}
%


\begin{document}

% --- Front matter ---

\frontmatter      % roman page numbering for front matter

\maketitle        % title page
\makeabstract     % abstract page

\tableofcontents  % table of contents

% --- Main matter ---

\mainmatter       % clear page, start arabic page numbering

\section{Topic Definition}

Developing software has become more and more working in group. One reason for this is agile methodologies like scrum and XP. Working in a group adds some dificulties for succeeding in project work. These dificulties can be seen as bad results in software.

There are many reasons why working as a group might effect on how project will achieve it's goals. One of those reasons is shared understanding. Shared understanding is related to picture that different stakeholders share between eachother during the life cycle of product development.

One way in trying to improve shared understanding in software projects is usage of different kinds of desing methodologies and mapping techniques. In this thesis work I will try to find answer for the question that does this kind of methods create better shared understanding between stakeholders in development group.

\section{Research Plan}


In this research I will use two groups. Both groups will design the solution for same software project. Both groups will consist fo students from department of Computer Science at University of Helsinki. Student subjects will be chosen randomly from the group of students that will participate in course software systems that will be held in autumn 2015.

Student subjects will be divided to groups randomly. After grouping has been done all subjects will answer five factor personality test\cite{fiveFactor}. This test will give me information about personalities and after the experiment I can compare results to allrady known personality questions in group work.

First group will be doing this design sesion without any specific methodology. Idea is trying to find people that don't have so much experience of designing of the software so there will be no great knoweledge about different desing patterns. Second group will be given knowledge about design methodology called User Story Mapping\cite{userStoryMapping}. This methodology will be thought to subjects by researcher.

After training session both groups will work with given problem and will try to design solution as best as they can. These session will be video recorded to gather more material about desing process and for evaluation of usega of desing methodology. When the desing process is finished and subjects will answer three different questionares that will give researcher tools to evaluate achieved shared understanding, created social networks and solution that was achieved.


\section{Previous Studies}

Humayun etc. studied effect of organizational structure in global software development (GSD) \cite{organizationalStructure}. Research question of the paper was set to "What is the impact of 'A clear organizational structure with communicating resposibilities' on shared understanding of requirements in GSD?". 

% --- References ---
%
% bibtex is used to generate the bibliography. The babplain style
% will generate numeric references (e.g. [1]) appropriate for theoretical
% computer science. If you need alphanumeric references (e.g [Tur90]), use
%
% \bibliographystyle{babalpha-lf}
%
% instead.

\bibliographystyle{babplain-lf}
\bibliography{references-en}


% --- Appendices ---

% uncomment the following

% \newpage
% \appendix
%
% \section{Example appendix}

\end{document}
