% --- Template for thesis / report with tktltiki2 class ---
%
% last updated 2013/02/15 for tkltiki2 v1.02

\documentclass[english]{tktltiki2}

% tktltiki2 automatically loads babel, so you can simply
% give the language parameter (e.g. finnish, swedish, english, british) as
% a parameter for the class: \documentclass[finnish]{tktltiki2}.
% The information on title and abstract is generated automatically depending on
% the language, see below if you need to change any of these manually.
%
% Class options:
% - grading                 -- Print labels for grading information on the front page.
% - disablelastpagecounter  -- Disables the automatic generation of page number information
%                              in the abstract. See also \numberofpagesinformation{} command below.
%
% The class also respects the following options of article class:
%   10pt, 11pt, 12pt, final, draft, oneside, twoside,
%   openright, openany, onecolumn, twocolumn, leqno, fleqn
%
% The default font size is 11pt. The paper size used is A4, other sizes are not supported.
%
% rubber: module pdftex

% --- General packages ---

\usepackage[utf8]{inputenc}
\usepackage[T1]{fontenc}
\usepackage{lmodern}
\usepackage{microtype}
\usepackage{amsfonts,amsmath,amssymb,amsthm,booktabs,color,enumitem,graphicx}
\usepackage[pdftex,hidelinks]{hyperref}

% Automatically set the PDF metadata fields
\makeatletter
\AtBeginDocument{\hypersetup{pdftitle = {\@title}, pdfauthor = {\@author}}}
\makeatother

% --- Language-related settings ---
%
% these should be modified according to your language

% babelbib for non-english bibliography using bibtex
\usepackage[fixlanguage]{babelbib}

% add bibliography to the table of contents
\usepackage[nottoc]{tocbibind}

% --- Theorem environment definitions ---

\newtheorem{thm}{Theorem}
\newtheorem{lem}[thm]{Lemma}
\newtheorem{cor}[thm]{Corollary}

\theoremstyle{definition}
\newtheorem{definition}[thm]{Definition}

\theoremstyle{remark}
\newtheorem*{remark}{Remark}


% --- tktltiki2 options ---
%
% The following commands define the information used to generate title and
% abstract pages. The following entries should be always specified:

\title{Design methods as tools for achieving shared understanding}
\author{Mika Kivi}
\date{\today}
\level{Master Thesis}
\abstract{Abstract.}

% The following can be used to specify keywords and classification of the paper:

\keywords{Software development, design, shared understanding, method}

% classification according to ACM Computing Classification System (http://www.acm.org/about/class/)
% This is probably mostly relevant for computer scientists
% uncomment the following; contents of \classification will be printed under the abstract with a title
% "ACM Computing Classification System (CCS):"
% \classification{}

% If the automatic page number counting is not working as desired in your case,
% uncomment the following to manually set the number of pages displayed in the abstract page:
%
% \numberofpagesinformation{16 pages + 10 appendix pages}
%
% If you are not a computer scientist, you will want to uncomment the following by hand and specify
% your department, faculty and subject by hand:
%
% \faculty{Faculty of Science}
% \department{Department of Computer Science}
% \subject{Computer Science}
%
% If you are not from the University of Helsinki, then you will most likely want to set these also:
%
% \university{University of Helsinki}
% \universitylong{HELSINGIN YLIOPISTO --- HELSINGFORS UNIVERSITET --- UNIVERSITY OF HELSINKI} % displayed on the top of the abstract page
% \city{Helsinki}
%


\begin{document}

% --- Front matter ---

\frontmatter      % roman page numbering for front matter

\maketitle        % title page
\makeabstract     % abstract page

\tableofcontents  % table of contents

% --- Main matter ---

\mainmatter       % clear page, start arabic page numbering

\section{Topic Definition}

Developing software and working in software projects is about group work. The role of group work has increased because of agile methodologies like Scrum\cite{scrum} and XP\cite{XP} that have achieved more space as project management techniques in software development. Group work brings its own challenges and benefits to the development process. Many software projects are large and dificult that it is impossible to finish them without group work.

Group work and skills that are required in succesfull group work have earned focus in research and practice. Companies have realiszed situtation that technical skills are not the only skills that effect on the success of software project. Having more knoweledge about soft skills companies can improve achieved results from the projects.

Focusing on improving group work is one way to tackle challenges that newer project management practices bring to the table. One of the ways of improvement is achieving better shared understanding between different stakeholders of the development group. Shared understanding is term that can be used to describe how widely individuals in project share common knoweledge about project requirements and project vision\cite{organizationalStructure}.

One way of trying to improve shared understanding in software projects is usage of different kinds of desing methodologies and mapping techniques. This thesis work will try to find answer for the question, does structured design methodologies have effect on achieved shared understanding? Achieved shared understanding is allways sum of many practicess and skills so there can't be straight answer for these question but what I can find out is that can shared understanding be improved with usage of structured design methdology.

\section{Research Methodology}

To examine shared understanding in project group level I will create experiment. In this part I will introduce my research methodology that I will be using in this research. Andrew J. Ko etal\cite{experiment} describes in their research challenges about experiment in software project research. In their research they also give some tips how researchers can tackle those problems. My research methdology is based on their paper and in this chapter I will describe it more detail.

\subsection{Recruitment}

First part of the planning and creating of the experiment is to find people to participate your experiment. This part will include planning of different kind of marketing materials\cite{experiment}. In my research I am not going to need real marketing materials because I am going to use university students from specific course. Still I need to create materials for students to courage them to participate to the experiment. For the students participating will give great opportunity to learn how to plan and design software development. Research will also show students results about their planning process.

\subsection{Selection}

Selection of the human subjects for the experiment is important part of experiment design. Selection criteria should be created real uses in mind\cite{experiment}. In my research I will use students as a subjects. Students will someday grow to software developers who will be using different kind of design tools.

In research I am planning to use students from course software project. In University of Helsinki this course is first time students will work in real software project that client has ordered. Using student from this course will effect my selection criteria. I do need to find two project groups that are working with project that are close to each other. With similar projects I can get more reliable results about differences between groups.

\subsection{Consent}

For the participants of the experiment reasearch working in acdemia is often required to explain ethical aspect of experiment\cite{experiment}. In my research I will use video recording to get more information about desing process for this I will need to promise from student subjects for filming.

\subsection{Procedure}

For the student subjects experiment will follow these steps. First chosen students will be answer two questionares. First questionare will collect the demographic information and promises to use collected data and video. Second questionare is personality questionare. I will use short five factor model to collet data about subjects personalities. I will use methodology described in this article\cite{fiveFactor}.

After the collection of the demographic data all subjects will be made familiar with data collection tools pathfinder\cite{pathfinder} and ACSMM\cite{acsmm}.This training will be done in one group and idea of training is to achieve certain level of knoweledge about used techniques. With that in mind researcher will create training session lasting approximately 30 minutes.

Next training session will be done with test group and I will call this group A for now on. Group A will be thought design methodology called User Story Mapping\cite{userStoryMapping}. This methodology is developed by Jeff Patton and it will be descibed more detail later in this paper. Researcher will use teaching example from the book for teaching methdology to student subjects.  

\subsection{Demographic measurement}

\subsection{Group assigment}

\subsection{Training}

\subsection{Tasks}

\subsection{Outcome measurement}

\subsection{Debrief and compensate}

In this research I will use two groups. Both groups will design the solution for same software project. Both groups will consist four students from department of Computer Science at University of Helsinki. Student subjects will be selected randomly from the group of students that will participate in course software systems that will be held in autumn 2015.

Student subjects will be divided to groups randomly. After grouping has been done all subjects will answer five factor personality model questionare\cite{fiveFactor}. This test will give me information about personalities and after the experiment I can compare results to allrady known personality questions in group work.

After the personality test both groups will do design phase of the given software project. Groupa A will be the group which will be given framework for design process. Framework used in this research is methodology User Story Mapping that is developed by Jeff Patton\cite{userStoryMapping}. User Story Mapping is chosen because it is new methdology and researcher are familiar with that methodology which make it easier to teach methdology to student subjects. There are plenty of others design frameworks that are used in software development and those could be also used in this research.

Group B is control group which will be working with no design methdologies. Student's are chosen from the course where they do not have much design and framework experience before experiment. In case where subjects are well knoweledged about design frameworks it might end up the case where control group will decide to use some methodology that group will know before hand. In the case where control group works with certain design methodology there is not way to find answer to research question which is looking for answer advantage of usege of design methodology.

After training session of group A both groups will work with given design problem and will design solution as best as they could. These design session will be video recorded to gather more information about used desing process and how different group dynamic . After finishing design process subjects will answer three different questionares that will give researcher tools to evaluate achieved shared understanding, created social networks and solution that was achieved in design process.


\section{Theoretical Background}

\subsection{Shared Understanding in Software Development}

Humayun etc. studied effect of organizational structure in global software development (GSD) \cite{organizationalStructure}. Research question of the paper was set to "What is the impact of 'A clear organizational structure with communicating resposibilities' on shared understanding of requirements in GSD?". From the findings of their studies we can conclude that clear organizational structure has clear improvement on achived shared understanding between different participants on development team.

Humayun etc. used concept mapping technique called Analysis Constructed Shared Mental Model (ACSMM) in measuring achived shared understanding between deifferent participants of experiment. In their experiment researchers had two different groups in study. One group did have clear organizational structure and communication protocol between different developers. Clear communication protocol had improving effect on Knoweledge Management (KM)\cite{organizationalStructure}.

\subsection{Skills Affecting on Group Work}

Software design process is more and more about group work and sharing knoweledge about project that is under development. Communication and understanding each others cultural background will affect on knoweledge will be trasported between others. In research paper Zarndt lists basic fundamentals how cultur and communication can be noticed in process\cite{culturalCommunication}. Respect is the most important aspect of clear communication in cross-culturall enviroment based on research paper. Other important factors in communication in these situations keeping it simple and repeating sayings. By repeating all participants can be sure that shared understanding between each other is achieved\cite{culturalCommunication}.

\subsection{Shared Understanding Research Methods}

There is several different methdos that can be used in investigating shared understanding and mental models of humans. Few examples of these methods are Analysis Constructed Shared Mental Model (ACSMM), Surface Matching and Deep Structure (SMD), Model Inspection Trace of Concepts and Relations (MITOCAR) and Dynamic Evaluation of Enhanced Problem-solving (DEEP). Each of these methods have it own strengths and weaknesess and they can be used in defferent occasions\cite{mentalModelResearch}.

\subsubsection{ACSMM}

ACSMM technique will change individual mental models to map that will show team sharedness and still keep details of inviduals vision. Technique is divided in five parts. First part is creating topic analyse that will help  individuals to make their own concept maps. Second and thrid phase of method is creating individual maps using Individually Constructed Mental Model (ICMM) elictation. After individual models has been created collected data will be used to determine which concept are shared between different individuals. Sharadness level shuold be decided to determine which is the level when concept is considered shared between memberes of the group. Johnson etc. said that good level to start is 50 percentage\cite{mentalModelResearch}. Last part of the model is creating shared mental model. This will include construction process in which different ICMM models are mapped together and result will show sharedness between those models. This technique can be used to evaluate sharedness over time and between different teams\cite{mentalModelResearch}.

\subsubsection{SMD}

SMD-technology uses three different instructional treatments. Data can be collected from concept maps or from natural language. Three treatments are named surface structure analysis, structurial properties analysis and deep structure analysis. SMD-technique can be used on determining team sharedness of change over time or between different teams\cite{mentalModelResearch}.

\subsubsection{MITOCAR}

Mitocar is based on software tool that uses natural language as input for model re-presentation. Mitocar creates graphical map for final results. In methodology there is few different steps in which program will use to parse naturall language to create the map. Drawing the map is done by calculating proximity vector from the data. Mitocar can be used in many different ways and with it there is possibility to compare models from different subject domains.

\subsubsection{DEEP}

Deep is methdology focusing on expert-like learning. This means that learning don't have single end point but it is evaluated as continiung process of growth. Deep uses persons ability to understand and solve complex problem spaces in general.

\section{Results}

\section{Discussion}

\section{Conclusions}

% --- References ---
%
% bibtex is used to generate the bibliography. The babplain style
% will generate numeric references (e.g. [1]) appropriate for theoretical
% computer science. If you need alphanumeric references (e.g [Tur90]), use
%
% \bibliographystyle{babalpha-lf}
%
% instead.

\bibliographystyle{babplain-lf}
\bibliography{references-en.bib}


% --- Appendices ---

% uncomment the following

% \newpage
% \appendix
%
% \section{Example appendix}

\end{document}
